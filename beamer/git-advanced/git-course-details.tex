\documentclass[aspectratio=169]{beamer}
%% For 4:3 aspect ratio:
%% \documentclass{beamer}
\usepackage{../git-course}

\title[git-course]{Git details}
\author{Chris Grandin \& Andy Edwards}
\date{\today}

\begin{document}

\frame[plain]{
\titlepage
}

\section{Introduction}
\frame{\frametitle{Commits and history}
  Git does not keep versions of software, it keeps \emph{commits}. The commits
  are kept track of using a hash code, using \emph{Secure Hash Algorithm 1}
  (SHA-1) which generates a hash key 40 digits long in hexadecimal. These are
  what you see on \gh\ and in various places when you use \gs.\\
  \bigskip
  Several of these commits have pointers to them which have special names:
  \bi
    \item the \textbf{HEAD} which points to the commit you are currently on
      in the \gs.
    \item the \textbf{master} which is just another branch and is the default
      one when you set up a repository on \gh.
    \item the branch names that you have in your repository
  \ei
  Once you've cloned a \gh\ repository, the \emph{master} points at the initial
  commit, and the \emph{HEAD} points at the master.
}

%% Figures from : https://marklodato.github.io/visual-git-guide/index-en.html?no-svg

\frame{\frametitle{Git layout}
  \begin{columns}
    \begin{column}{0.4\linewidth}
      \bi
        \item \textcolor{green}{Green} items are commits.
        \item \textcolor{orange}{Orange} items are branches.
        \item The current branch is always pointed to by \textbf{HEAD}.
        \item \textbf{ed489} is the most recent commit, and it is pointed to
          by \textbf{master} and \textbf{HEAD}.
        \item \textbf{maint} is another branch, and is an ancestor of
          \textbf{master}.
      \ei
    \end{column}
    \begin{column}{0.6\linewidth}
      \includegraphics[
        width=\textwidth,
        height=0.8\textheight,
        keepaspectratio]
                      {figures/head-master}
    \end{column}
  \end{columns}
}

\frame{\frametitle{Commit on master}
  \includegraphics[
    width=\textwidth,
    height=0.8\textheight,
    keepaspectratio]
    {figures/commit-master}
}

\frame{\frametitle{Commit on another branch}
  \includegraphics[
    width=\textwidth,
    height=0.8\textheight,
    keepaspectratio]
    {figures/commit-maint}
}

\frame{\frametitle{Advanced stashing}
  You can stash more than one set of changes if you wish. To see your list
  of available stashes:\\
  \lstinline{git stash list}\\
  The list will look something like this:\\
  \lstinline{stash@\{0\}: WIP on master: 46ae7a8 Merge conflict resolved.}\\
  \lstinline{stash@\{1\}: WIP on master: fe35a9b Minor change to Readme.}\\
  \lstinline{...}\\
  where each commit the stash was based on is listed. You can apply any
  of these by using the syntax:\\
  \lstinline{git stash apply stash@\{N\}}\\
  where N is one of the numbers shown. If you don't specify a stash, the
  most recent one is used, as shown in the previous slide.
}

\end{document}
