\documentclass[aspectratio=169]{beamer}
%% For 4:3 aspect ratio:
%% \documentclass{beamer}
\usepackage{../git-course}

\title[git-course]{Motivation for using \gh}
\author{Andrew Edwards \& Chris Grandin}
\date{\today}

\begin{document}
%% Needed to remove 'Figure:' from figure captions:
\setbeamertemplate{caption}{\raggedright\insertcaption\par}

\frame[plain]{
\titlepage
}

\section{Introduction}

\frame{\frametitle{Motivation}
  \bi
    \item We are working far more collaboratively than in the past -- sharing code and writing documents.
    \item Stock assessments, for example, can be extremely complex with Bayesian output (recent Pacific Ocean Perch assessment had $>200$ figures in just one Appendix).
    \item How to share code and make sure we are working on the same version?
    \item Emailing versions of files back and forth is:
    \bi 
      \item very inefficient,
      \item prone to errors,
      \item painful.
    \ei
    \item With complex code we need to have {\red identical} folder structures on each other's computers. 
  \ei
}

\section{Avoid}
\frame{\frametitle{Examples of what we can avoid}

  \centering
  \begin{figure}
    \includegraphics[
        width=\textwidth,
        height=0.8\textheight,
        keepaspectratio]
        {figures/interim.png}
  %\vspace{-3mm}
  %\caption{\tiny\url{https://www.reddit.com/user/NegativePitch}}
  \end{figure}
}

\frame{\frametitle{Relying on one person (e.g.~me) who holds up a project}

  Often we may be collaborating on a project but be busy on something else, yet our collaborator has time.

  \centering
  \begin{figure}
    \includegraphics[
        width=\textwidth,
        height=0.8\textheight,
        keepaspectratio]
        {figures/procEmail.png}
  %\vspace{-3mm}
  %\caption{\tiny\url{https://www.reddit.com/user/NegativePitch}}
  \end{figure}
}

\frame{\frametitle{Cluttering up directories}

  We may want to keep old versions in case we have to go back, but then ...

  \centering
  \begin{figure}
    \includegraphics[
        width=\textwidth,
        height=0.8\textheight,
        keepaspectratio]
        {figures/EAversions.png}
  \end{figure}
}

\frame{\frametitle{Can work simultaneously on same document (and then merge)}

  Can all keep up with the latest version (checking and merging each 
  other's work along the way), rather than ...

  \centering
  \begin{figure}
    \includegraphics[
        width=\textwidth,
        height=0.8\textheight,
        keepaspectratio]
        {figures/proposalEmail.png}
  \end{figure}
}

\frame{\frametitle{Can then avoid very annoying situations}

  Can all keep up with the latest version (checking and merging each 
  other's work along the way), rather than ...

  \centering
  \begin{figure}
    \includegraphics[
        width=\textwidth,
        height=0.8\textheight,
        keepaspectratio]
        {figures/dogsBreakfast.png}
  \end{figure}
}



\section{\gh\ advantages}

\frame{\frametitle{Can easily catch up with collaborator}

  Chris did a lot of work (`commits') from 11th May ...

  ~\\
  ~\\
  \includegraphics[
     width=\textwidth,
     height=0.6\textheight,
     keepaspectratio]
     {figures/netWorkGraph1}
}

\frame{\frametitle{Can easily catch up with collaborator}

  ... to 30th May:
  ~\\
  ~\\

  \includegraphics[
     width=\textwidth,
     height=0.6\textheight,
     keepaspectratio]
     {figures/netWorkGraph2}

  I did nothing in that time ... 

}

\frame{\frametitle{Can easily catch up with collaborator}

  ... but just needed three commands (that you'll learn) to get caught up:
  ~\\
  ~\\

  \includegraphics[
     width=\textwidth,
     height=0.6\textheight,
     keepaspectratio]
     {figures/netWorkGraph3}

  ~\\
  All files that Chris edited got updated, plus I now have any new files he created, and our folder structures are identical. 
}

\frame{\frametitle{Easy to answer other's questions}

  \includegraphics[
     width=\textwidth,
     height=0.80\textheight,
     keepaspectratio]
     {figures/MEEcodeQuest}


  Rather than go to the code on my laptop that I haven't looked at for six months, I could click on link and answer very quickly.
}

\frame{\frametitle{Easy to answer other's questions}

  \includegraphics[
     width=\textwidth,
     height=0.80\textheight,
     keepaspectratio]
     {figures/MEEcodeAnswer}

~\\
  Can provide a clear answer with a link to the file I'm talking about. No ambiguity.
}



\frame{\frametitle{I wanted to check if I'd changed anything in the last month}

  \includegraphics[
     width=\textwidth,
     height=0.80\textheight,
     keepaspectratio]
     {figures/history}

  I hadn't thought of that before [on \gh: Code -- $<$click file$>$ -- History].
}


\frame{\frametitle{Hake assessment}

  \bi 
    \item Collaboration between four/five US and Canadian scientists.
    \item Annual assessment.
    \item Full Bayesian statistical catch-at-age model (complex output).
    \item Short turnaround between getting final data and submitting assessment.
    \item Short turnaround between review meeting and publishing assessment (two days).
    \item 2015 and before -- lots of editing and amalgamating Word files late at night. 
  \ei
}

\frame{\frametitle{Hake assessment}
  \bi
    \item After running models we can now automatically generate full document, including all numbers, figures and tables (via knitr and latex, all shared on \gh).

    \item Was a lot of work in 2016 to set up, but 2017 was way less stressful and resulted in more polished document.
    \item Praise on Twitter from one of the reviewers!      
  \ei
  \includegraphics[
     width=\textwidth,
     height=0.50\textheight,
     keepaspectratio]
     {figures/trevorTweet}
}

\frame{\frametitle{Who wrote that piece of garbage?}

  \includegraphics[
     width=\textwidth,
     height=0.80\textheight,
     keepaspectratio]
     {figures/blame}

~\\
  Can be useful [on \gh: Code -- $<$click file$>$ -- Blame].
}

\frame{\frametitle{Can properly keep track of (and discuss) issues}

Rather than lots of emails that get forgotten, the to-do list actually gets completed.

  \includegraphics[
     width=\textwidth,
     height=0.80\textheight,
     keepaspectratio]
     {figures/hakeIssues}
}

\frame{\frametitle{Can properly keep track of (and discuss) issues}

And you don't have to follow up with co-authors (who do have it under control):

~\\
  \includegraphics[
     width=\textwidth,
     height=0.80\textheight,
     keepaspectratio]
     {figures/captionEmail}
}


\section{Summary}

\frame{\frametitle{Summary}

  \bi
    \item Ideal for sharing of code.
    \item `Repositories' can be public or private.
    \item Caveat -- cannot (properly) collaborate on Word files. We will concentrate at first on just sharing code. 
    \item BUT there's a way of converting Markdown using pandoc to generate a Word file (Sean to give a short demo tomorrow). So you don't have to learn \LaTeX.
    \item Hake example is to show how far you can get.
    \item There is a learning curve, but once you use \gh\ you only really need a few main commands.
  \ei
}



\end{document}
